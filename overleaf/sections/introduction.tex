As expressões faciais são um componente não-verbal da comunicação humana, compreendendo sorrisos, franzir de sobrancelhas e outras ações, com vistas a transmitir emoções e sentimentos \cite{Ekman1971}. Por meio do reconhecimento e interpretação de expressões faciais é possível obter um \emph{feedback} mais verossímil e imediato da percepção dos indivíduos sobre um determinado contexto ou situação, razão pela qual é adotada, por exemplo, para avaliação do efeito de fármacos, detecção de fadiga de motoristas e pilotos, etc. \cite{Fasel2003}. Dada a importância desta percepção, é natural a demanda pelo desenvolvimento de soluções automáticas que capturem o significado das expressões faciais de indivíduos.

Nesta perspectiva, a classificação automática de expressões faciais consiste em localizar faces humanas em uma cena, extrair características faciais da região detectada, reconhecer padrões e então categorizar o resultado obtido em uma das sete expressões faciais canônicas: felicidade, tristeza, medo, nojo, surpresa, raiva e neutro \cite{Pantic2009fea}. Esta tarefa, entretanto, envolve a superação de diversos desafios, tais como as distintas formas que estas expressões podem ser manifestadas por indivíduos diferentes, a presença de outros elementos corporais, como as mãos, para composição da expressão, e até mesmo as características pessoais dos sujeitos, como presença ou ausência de barba, óculos, etc.

Com o advento dos métodos de \emph{Machine Learning}, houve um progresso na classificação automática de expressões faciais. As técnicas de \emph{Deep Learning}, em particular, têm colaborado para o avanço do estado da arte neste problema, este sucesso está relacionado a alta habilidade de representação das informações de entrada, utilizando várias camadas de neurônios artificiais compondo vários níveis de abstração\cite{lecun2015deep}. Porém, mais esforços precisam ser efetuados para uma melhor eficiência nesta tarefa.

Considerando o contexto apresentado, este trabalho se propõe a apresentar uma abordagem baseada no uso de Redes Neurais Convolucionais organizadas segundo um \emph{ensemble} com votação mediada por um modelo de \emph{Machine Learning} baseado em \emph{Boosting}, para determinação das expressões correspondentes a partir de imagens de faces humanas. Os resultados obtidos com esta abordagem mostram-se animadores, pois obteve-se um desempenho de 71.74\% nesta tarefa, equiparável com algumas das contribuições mais recentes da literatura neste problema.

Para apresentar os resultados obtidos, este trabalho está organizado como segue. A Seção X apresenta uma visão geral do estado da arte para a classificação automática de expressões faciais humanas com técnicas de \emph{Deep Learning}. Em seguida, na Seção Y, são apresentados os materiais e métodos a serem considerados na elaboração da solução proposta. Os resultados obtidos e a discussão são então apresentados na Seção Z. Por fim, as considerações finais e perspectivas de trabalhos futuros são mostradas na Seção W. \todo{Completar}
