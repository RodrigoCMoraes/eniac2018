%https://ibug.doc.ic.ac.uk/media/uploads/documents/EncycBiometrics-Pantic-FacExpRec-PROOF.pdf
A Classificação de Expressões Faciais é um processo executado por humanos e computadores que consiste em localizar faces em uma cena, extrair características faciais da região detectada, analisar alterações das características faciais como um sorriso ou um franzir de sobracelhas, e categorizar o resultado em uma expressão como felicidade ou raiva, por exemplo \cite{Pantic2009fea}.

Tecnologias que podem interpretar e responder de forma automática a expressões faciais já encontram uma grande variedade de aplicações, dada sua importância social. Exemplos disto, são sistemas de ensino que utilizam a expressão facial dos alunos como \textit{feedback}, teste da efetividade de fármacos anti-depressivos e detecção de fadiga de motoristas e pilotos \cite{Fasel2003}.

Graças a introdução de métodos de \textit{Machine Learning}, tem-se  avançado no campo de Classificação Automática de Expressões Faciais. Mais especificamente, métodos de \textit{Deep Learning} tem apresentado resultados bons nas tarefas que envolvem o uso de detecção de padrões e extração de características em imagens, nas mais variadas situações e contextos \cite{whitehill2013automatic}.

Paul Ekman e Friesen postularão seis emoções primárias que possuem cada uma conteúdo próprio e associação a uma única expressão facial. Estas emoções se monstram invariantes ao longo das diversas culturas humanas e são identificadas como felicidade, tristeza, medo, nojo, surpresa e raiva \cite{Ekman1971}.

O presente trabalho apresentou resultados equiparáveis ao da literatura na tarefa de Classificação Automática de Expressões Faciais nas expressões primárias, utilizando modificação não-trivial, votação mediada por um modelo inteligente, em \textit{Emsemble} de Redes Neurais Convolucionais.
