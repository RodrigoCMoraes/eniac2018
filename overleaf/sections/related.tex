No campo da visão computacional a pesquisa de Classificação Automática de Expressões Faciais tem se mostrado bastante ativa. Muitos trabalhos tem aplicado os modelos de Redes Convolucionais Neurais na tarefa de Classificação de Expressões Faciais, o que pode ser observado pelos trabalhos relacionados, em que todos utilizam de alguma forma uma combinação com o modelo Convolucional, bem como o \textit{Ensemble} destes. Além disto, a construção destes modelos seguem padrão de construção semelhante ao deste trabalho, descrito na Subseção 3.3, diferenciando-se apenas na última camada, Camada de Saída, por não fazerem uso da função \textit{Softmax}, mas sim das especificidades de sua arquitetura.

A utilização de técnicas de pré-processamento bem como o emprego de filtros comumente utilizados na visão computacional para comparação de imagens, tem-se mostrado favoráveis no aumento do desempenho de modelos classificadores de expressões faciais.  Exemplo de técnicas de pré-processamento pode ser visto em \cite{Kim2016FusingAA}, onde é analisado o alinhamento da face contida na imagem, e mostrado que a correção desta condição se apresenta favorável ao desempenho de dectores automáticos.  E exemplo de filtros de visão computacional podem ser vistos em \cite{al2016facial}, onde é analisado o uso combindado de SIFT, um descritor de imagem para correspondência e reconhecimento de pontos de interesse, e sua variação D-SIFT \cite{lindeberg2012scale}. Contudo, \cite{DBLP:journals/corr/PramerdorferK16} apresenta e evidencia que resultados competitivos com o estado-da-arte podem ser obtidos mesmo sem o emprego de pré-processamento e filtros de computacional sofisticados, mas também que modelos razos combinados ou não, para esta tarefa em específico, apresentam resultados igualmente competitivos. Estas comprovações são ainda reforçadas por \cite{DBLP:journals/corr/Tang13}, o ganhador da competição em que a base de dados foi empregada publicamente pela primeira vez. Onde este não utilizou \textit{Ensemble} de modelos de Redes Convolucionais, mas sim, uma única rede que possui na saída um modelo SVM Linear, onde comumente são empregados camadas \textit{Fully Connected}.

Diferentemente das pesquisas anteriores, este trabalho utiliza nas Redes Convolucionais o mesmo padrão de arquitetura, bem como camadas de saída \textit{Fully Connected} e função de ativação \textit{Softmax} na última. Além disto, o \textit{Ensemble} deste é realizado pelo modelo \textit{XGBoost}, que é responsável por analisar e combinar os resultados para então classificar a entrada.
