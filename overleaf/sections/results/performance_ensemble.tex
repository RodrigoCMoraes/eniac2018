O método utilizado para obtenção dos valores de Micro \textit{F1 Score} e acurácia para o \textit{Ensemble} foi o mesmo utilizado para os modelos de Rede Convolucional e \textit{Xtreme Gradient Boosting}. Ressaltando que os resultados das medidas de desempenho obtiveram os mesmos valores, sendo este o motivo de serem apresentados somente os valores da medida Micro \textit{F1 Score} na Tabela \ref{tbl:fscore}.

\begin{table}[!htb]
\centering
\caption{F1 Micro das Arquiteturas utilizadas}
\label{tbl:fscore}
\begin{tabular}{@{}cc@{}}
\toprule
Modelo & F1 Micro           \\ \midrule
1      & 0.6898857620507105 \\
2      & 0.6767901922541097 \\
3      & 0.6606297018668152 \\
4      & 0.6798551128448036 \\
5      & 0.6667595430482028 \\
6      & 0.6781833379771524 \\
7      & 0.694901086653664  \\
8      & 0.6285873502368348 \\
9      & 0.6244079130677069 \\ 
ensemble & 0.7174700473669546 \\ \bottomrule
\end{tabular}
\end{table}

Como pode ser visto na Tabela \ref{tbl:fscore}, todos os modelos obtiveram bons resultados nesta tarefa de classificação, pois, considerando-se um palpite aleatório há $14.28$\% de chances de acerto de uma classificação correta, levando em consideração que $7$ é a quantidade de classes desta tarefa, e que os modelos obtiveram pelo menos $62.44$\% como valor de Micro \textit{F1 Score}. Apesar disto, alguns modelos se destacam, como são os casos dos modelos 7 e 1, com valores de $69.49$\% e $68.98$\% respectivamente, na métrica de desempenho.

Além disto, os modelos 7 e 1 também se destacam devido a serem os mais profundos e por terem as maiores quantidades de parâmetros treináveis, 6.653.599 e 15.250.887 respectivamente. Juntos, estes correspondem a $66.91$\% do total de parâmetros treináveis do \textit{Ensemble}, que é de $32.738.871$. Observando esta relação, valor de Micro \textit{F1 Score} com quantidade de parâmetros treináveis, vê-se que os modelos mais profundos, consequentemente com maior número de parâmetros treináveis, foram os que obtiveram melhores desempenho na classificação geral. Enquanto os modelos mais rasos, apesar de obterem desempenho abaixo dos profundos, obtiveram os melhores resultados em classificar expressões específicas.\todo{Como demonstrar isso?}

Outro fator interessante de ser observado em relação a quantidade de parâmetros é a ordem de grandeza do modelo responsável pelo sistema de votação. Que possui valor na ordem de algumas dezenas de milhares, enquanto os modelos de extração de características possuem valores na ordem de algumas dezenas de milhões. Este comportamento se mostra coerente, levando em consideração que os tipos das tarefas realizadas por cada modelo possuem complexidades bastante distintas, os modelos convolucionais responsáveis pela extração das várias características que compoem uma expressão facial humana, enquanto o modelo da votação combina da melhor forma possível, dados os parâmetros, as respostas dos modelos convolucionais.
