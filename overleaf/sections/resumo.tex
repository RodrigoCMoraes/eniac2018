\begin{resumo}

A Expressão Facial é um fator de suma importância na interação social dos seres humanos. E tecnologias que podem interpretar e responder de forma automática a estímulos de expressões faciais já encontram uma grande variedade de aplicações, desde teste de fármacos anti-depressivos, até análise de fadiga de motoristas e pilotos. Neste contexto, o seguinte trabalho apresenta um modelo para Classificação Automática de Expressão Facial utilizando como base de treinamento o \textit{dataset} \textit{Challenges in Representation Learning: Facial Expression Recognition Challenge}(FER2013), caracterizado por exemplos de expressões faciais espontâneas em ambientes não controlados.  O método apresentado é composto por uma arquitetura \textit{Ensemble} de Redes Convolucionais Neurais, utilizando um sistema de votação não-trivial, baseado em um modelo inteligente, \textit{Xtreme Gradient Boosting} - \textit{XGBoost}. Como critérios de desempenho para validação do modelo proposto foram empregadas técnicas de K-fold e F1 Score Micro, para garantia de robustes e confiança dos resultados, que são competitivos com trabalhos estado-da-arte.

\end{resumo}
